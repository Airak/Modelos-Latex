\documentclass[a4paper, 12pt, oneside]{abntex2}
	\usepackage[T1]{fontenc}
	\usepackage[utf8]{inputenc}
	\usepackage[brazil]{babel}
	\usepackage{tikz}
	\usepackage{lmodern}
	\usepackage{amsmath}
	\usepackage{titling}
	\usepackage[margin = 2cm]{geometry}
	\usepackage{amsthm}
	\usepackage{abntex2cite}
	\usepackage[lined, ruled, portuguese]{algorithm2e}
	\usepackage{float}
	\usepackage{listings}
	\usepackage{color}
	\usepackage{amsmath}
	\usepackage{longtable}
	\usepackage{verbatim}
	\usepackage{multirow}
	
\definecolor{mygreen}{rgb}{0,0.6,0}
\definecolor{mygray}{rgb}{0.5,0.5,0.5}
\definecolor{mymauve}{rgb}{0.58,0,0.82}

\lstset{ %
	backgroundcolor = \color{white},   % choose the background color; you must add \usepackage{color} or \usepackage{xcolor}
	basicstyle = \footnotesize\ttfamily,        % the size of the fonts that are used for the code
	breakatwhitespace = true,         % sets if automatic breaks should only happen at whitespace
	breaklines = true,                 % sets automatic line breaking
	captionpos = b,                    % sets the caption-position to bottom
	commentstyle = \color{mygreen},    % comment style
%	deletekeywords = {...},            % if you want to delete keywords from the given language
	escapeinside = {\%*}{*)},          % if you want to add LaTeX within your code
	extendedchars = true,              % lets you use non-ASCII characters; for 8-bits encodings only, does not work with UTF-8
	frame = false,	                   % adds a frame around the code
	keepspaces = true,                 % keeps spaces in text, useful for keeping indentation of code (possibly needs columns=flexible)
	keywordstyle = \color{blue},       % keyword style
%	otherkeywords = {*,...},           % if you want to add more keywords to the set
	numbers = left,                    % where to put the line-numbers; possible values are (none, left, right)
	numbersep = 5pt,                   % how far the line-numbers are from the code
	numberstyle = \tiny\color{mygray}, % the style that is used for the line-numbers
	rulecolor = \color{black},         % if not set, the frame-color may be changed on line-breaks within not-black text (e.g. comments (green here))
	showspaces = false,                % show spaces everywhere adding particular underscores; it overrides 'showstringspaces'
	showstringspaces = false,          % underline spaces within strings only
	showtabs = false,                  % show tabs within strings adding particular underscores
	stepnumber = 2,                    % the step between two line-numbers. If it's 1, each line will be numbered
	stringstyle = \color{mymauve},     % string literal style
	tabsize = 2,	                   % sets default tabsize to 2 spaces
	title = \lstname                   % show the filename of files included with \lstinputlisting; also try caption instead of title
}



\SetKwBlock{Inicio}{Início}{Fim}

\autor{Rodrigo Rodrigues de Novaes Júnior}

\titulo{Inteligência Computacional}
	
\instituicao{Centro Federal de Educação Tecnológica de Minas Gerais
	\par
	Curso de Engenharia de Computação
	\par
	DECOM -- Departamento de Computação}

\data{\today}

\orientador{Prof. Dr. Rogério Martins Gomes}

\renewcommand{\maketitle}{
	\begin{center}
		\imprimirtitulo \\
	\end{center}
	
	\begin{tabular}{lr}
		Professor: \imprimirorientador.
		
		&Aluno: \imprimirautor.\\
	\end{tabular}
	
	\vspace{.4em}
	\noindent
	Terceira lista de exercícios, \imprimirdata.\\
}

\newtheoremstyle{questao_estilo}% <name>
	{3pt}			% <Space above>
	{3pt}			% <Space below>
	{}				% <Body font>
	{}				% <Indent amount>
	{\bfseries}		% <Theorem head font>
	{~--}			% <Punctuation after theorem head>
	{.5em}			% <Space after theorem headi>
	{}				% <Theorem head spec (can be left empty, meaning `normal')>

\theoremstyle{questao_estilo}
\newtheorem{questao}{Questão}

\begin{document}
	\maketitle
	
	\noindent\rule{\textwidth}{0.5mm}
\textbf{Questão 04}\\
\rule{\textwidth}{0.5mm}

O método Elbow (cotovelo) é bastante utilizado para essa finalidade. Consiste em testar os possíveis valores de $K$, partindo de 1, até que a variação da função custo seja tão pequena que possa ser desprezada. Isso faz com que exista um ponto, conhecido como ``cotovelo'', cuja variação para o ponto anterior é elevada, mas pequena em relação ao sucessor. O método indica que esse valor seja adequado para atribuir a $K$.\\

Outra estratégia plausível é factível busca binária com base na minimização da função custo. Se escolhermos um intervalo $K \in [1, L)$ tal que a função custo $J(\ldots)_K$ seja estritamente decrescente, pode-se executar o seguinte algoritmo:

\begin{equation*}
    find(l_i, l_f) = \left\{
    \begin{array}{ll}
        find\left(l_i, \dfrac{l_i + l_f}{2} \right), &\textrm{se $J(\ldots)_{(l_i + l_f)/2)} \geq J(\ldots)_{l_i}$}\\
        find\left(\dfrac{l_i + l_f}{2}, l_f \right), &\textrm{se $J(\ldots)_{(l_i + l_f)/2)} < J(\ldots)_{l_i}$}\\
        \dfrac{l_i + l_f}{2},    &   \textrm{se $l_i = l_f + 1$},\\
    \end{array}
    \right.
\end{equation*}

\noindent
onde $J(\ldots)_K$ é a função custo avaliada para um valor de $K$, $l_i$ o valor inferior e $l_f$ o valor superior de um subintervalo pertencente a $[1, L)$, tal que para todo $[l_i, l_f) \subseteq [1, L)$, o valor de $K$ deve estar contido em $[l_i, l_f)$. Nessas condições, $K = find(1, L)$ fará com que $J(\ldots)_K$ seja mínimo.\\

\noindent\rule{\textwidth}{0.5mm}
\textbf{Questão 06}\\
\rule{\textwidth}{0.5mm}

Seja $x_1$ o vetor que representa a entrada em ml/dia de café e $x_2$ a entrada em ml/dia de leite, sabendo que precisamos gerar $K = 3$ perfis de pessoas, bem como tendo os seguintes centroides iniciais:

\begin{equation*}
    C_1(10, 30); C_2(45, 46); C_3(55, 57),
\end{equation*}

\noindent
precisamos utilizar a distância euclidiana, dada por

\begin{equation}
    d((x_a, y_a), (x_b, y_b)) = \sqrt{(x_a - x_b)^2 + (y_a - y_b)^2}
\end{equation}

\noindent
para definir as novas posições de $C_1$, $C_2$ e $C_3$ a partir das médias de todo $p_j \in P$, $P = [ x_1 x_2 ]$ sendo a matriz de entrada, onde $i$ representa um dos agrupamentos. Nesse contexto, a seguinte tabela apresenta o agrupamento de uma entrada e a distância euclidiana para seu centroide correspondente:

\begin{center}
    \begin{tabular}{lll}
        \hline
        $i$                 & $p_j$     &   $d(C_i, p_j)$   \\     
        \hline
        1                   & (32, 27)  &   22.203603       \\
        2                   & (55, 43)  &   10.440307       \\
        3                   & (80, 63)  &   25.709920       \\
        3                   & (85, 50)  &   30.805844       \\
        2                   & (58, 38)  &   15.264338       \\
        3                   & (82, 55)  &   27.073973       \\
        1                   & (25, 31)  &   15.033296       \\
        3                   & (66, 42)  &   18.601075       \\
        3                   & (60, 49)  &   9.433981        \\
        1                   & (35, 12)  &   30.805844       \\
         \hline
    \end{tabular}
\end{center}

Isso responde ao item \textbf{a)} da questão. Para o item \textbf{b)}, define-se um novo centroide por

\begin{equation*}
    C_{i}'\left( \dfrac{1}{|i|}  \sum_{j \in C_{i}} x_{1j}, x_{2j}) \right).
\end{equation*}

\noindent
onde $|i|$ é o número de elementos no agrupamento $i$. Portanto, os novos centroides são:

\begin{equation*}
    C_1(30.67, 23.33); C_2(56.50, 40.50); C_3(74.60, 51.80).
\end{equation*}

Ao recalcular o agrupamento de cada $p_j$, temos:

\begin{center}
    \begin{tabular}{lll}
        \hline
        $i$                 & $p_j$     &   $d(C_i, p_j)$   \\     
        \hline
        1                   & (32, 27)  &   3.903562        \\
        2                   & (55, 43)  &   2.915476        \\
        3                   & (80, 63)  &   12.433825       \\
        3                   & (85, 50)  &   10.554620       \\
        2                   & (58, 38)  &   2.915476        \\
        3                   & (82, 55)  &   2.915476        \\
        1                   & (25, 31)  &   9.538228        \\
        2*                  & (66, 42)  &   9.617692        \\
        2*                  & (60, 49)  &   9.192388        \\
        1                   & (35, 12)  &   12.129213       \\
         \hline
    \end{tabular}
\end{center}

Todas as linhas marcadas com * apresentaram mudanças de agrupamento. A tabela final é mostrada acima.\\

\noindent\rule{\textwidth}{0.5mm}
\textbf{Questão 07}\\
\rule{\textwidth}{0.5mm}

Queremos calcular

\begin{equation*}
    \Sigma = \dfrac{1}{m} \sum_{i = 1}^{m} \left( x^{(i)} \right) \left({x^{(i)}} \right)^T,
\end{equation*}

\noindent
onde $m$ é a dimensão da entrada e $A'$ é a transposta de $A$.

%\textbf{Extras para estudo:}

%\begin{algorithm}
%    \caption{Algoritmo $k$-means, com inicialização aleatória.}
%    \Inicio {
%    
%    }
%\end{algorithm}
\end{document}