    \noindent\textbf{TEXTO:}\\
    
    \noindent LIMA, M. E. A., ASSUNÇÃO, A. A. \& FRANCISCO, J. M. S. D. (2002). Aprisionado pelos ponteiros de um relógio - um caso de transtorno mental desencadeado no trabalho. In W. Codo \& M. G. Jacques (Orgs.), Saúde Mental e Trabalho – Leituras (pp. 209-246). Petrópolis: Vozes 

    \begin{description}
    
        \item \textbf{Questão 01} -  Descreva as características do modo como Carlos foi criado por seus pais (valores, regras, etc). \\
        
        Os pais de Carlos se preocupavam bastante com os filhos sendo capazes de dar um bom exemplo e educá-los da forma devida, embora não possuíssem condições financeiras suficientes para fornecer todo o conforto que desejavam dar aos filhos. Desta forma Carlos foi ensinado a ser um homem responsável, honesto e dedicado ao trabalho, além de disciplinado e a manter a integridade de seu nome. \\
        
        O pai era bastante rigoroso (mas não era violento) e só permitia que os filhos saíssem com pessoas de confiança, por conta de todo o cuidado dos pais, Carlos não passeava muito, quando o fazia provavelmente era acompanhado de algum de seus irmãos. Apesar disso tudo ele não se sentia preso, encarava aquilo com bastante naturalidade. \\
        
        Os filhos nunca foram obrigados a trabalhar, mas faziam isso por gostar e por encarar como sua responsabilidade, sendo assim Carlos sempre foi bastante trabalhador e dedicado ao serviço, tendo relatos de experiências positivas em praticamente todos os trabalhos. \\
        
        Apesar das condições financeiras da família, os pais sempre forneciam aos filhos aquilo que podiam, por conta disso Carlos conseguiu estudar até a quarta série, visto que após isso seria necessário um gasto maior para se locomover para cidade vizinha para estudar. Os colegas e professores de Carlos gostavam bastante dele, que era um jovem estudioso e trabalhador, sendo que nunca ficava vadiando na rua.
                
        \item \textbf{Questão 02} - Enumere as características que Carlos valorizava enquanto profissional (nele e nos demais). 
        
        \begin{enumerate}
            \item Carlos era trabalhador. Quando trabalhava na roça ele era o primeiro a sair de casa e o último a chegar do trabalho, trabalhando mais que o duas ou três vezes mais do que os outros.
            \item Era bastante dedicado ao trabalho, em seu último trabalho chegou a tirar férias apenas uma vez, além de ser bastante pontual.
            \item Gostava de receber oportunidades e lidava bem com o depósito de confiança e responsabilidades em cima de sua pessoa. 
            \item Não gostava de trabalhadores que não sabiam agir com respeito, por isso era bastante calmo e procurava conversar de uma forma que não magoasse os outros, sendo educado e cortês.   
            \item Valorizava a disciplinas no ambiente de trabalho, procurando sempre conversar e aconselhar aqueles que infligiam as regras, como os que deixavam de acionar o relógio.
            
        \end{enumerate}
        
        \item \textbf{Questão 03} - Aponte os elementos presentes em seu último emprego que podem ter contribuído para a deterioração de sua saúde e os sintomas por ele desenvolvidos relacionados. 
        
        \begin{itemize}
            \item Os mecanismos de controle, sendo a câmera e o relógio os principais, visto que aumentam a tensão no trabalho. A tensão produzida em Carlos era tamanha que este acionava o relógio mesmo quando ele estava desligado e sentia que estava sendo vigiado em todos os momentos, mesmo que a câmera estivesse ligada apenas até as 21 horas, além de não permitir que ele fosse ao banheiro com a frequência desejada.
            \item Os fortes riscos e a responsabilidade envolvida no trabalho, uma vez que poderia acontecer um acidente caso ele fosse imprudentes no trabalho, o que também aumentava a tensão.
            \item A eliminação de fontes de distração no trabalho, como a proibição de conversas e a leitura de revistas e jornais, o que era agravado pelo turno em que o trabalho era exercido. O que poderia aumentar o sono devido a monotonia e a fadiga, para combater isso, Carlos tomava café e começou a fumar. 
            \item Os conflitos existentes na relação entre o chefe e seus subordinados e outros no meio de trabalho. O síndico era considerado um chefe autoritário e capaz de humilhar os empregados até mesmo na presença de terceiros no serviço. 
            \item O turno noturno de trabalho, uma vez que  a maioria das pessoas tem suas funções físicas orientadas para atividades diurnas, dedicando a noite ao descanso. Gerando distúrbios de sono, como insônia por exemplo, além de alterações gastrointestinais e reforçar ainda mais o isolamento proporcionado em seu ambiente de trabalho.
            \item As buzinas e campainhas, que o assustavam quando eram acionadas, mas o deixavam sobressaltado quando não, uma vez que um usuário poderia ter deixado de tocá-la podendo acarretar em um acidente.
            \item O medo do desemprego que reforçava os impactos negativos do trabalho.
            \item Foram retirados alguns benefícios como o lanche, desta forma Carlos começou a trazer o lanche da própria casa. 
            \item Carlos passou a beber para conseguir dormir durante as folgas. 
            \item Quando estava nervoso Carlos sentia dores de estômago e de barriga além da canseira. Após ser internado teve de começar a tomar remédios de pressão e para combater a insônia. 
        \end{itemize}
        
        \item \textbf{Questão 04} - Dê sua opinião sobre a questão levantada no início do texto: “o trabalho provoca ou precipita transtornos mentais?”. \\
        
        Como vimos através do texto, Carlos continha um aguçado senso de responsabilidades, comportamento disciplinado e uma grande dedicação ao trabalho, características herdadas através de sua criação e que sempre o ajudaram a se destacar positivamente em suas outras experiências de trabalho, mas que apenas agravaram os fatores prejudiciais que existiam em seu último ambiente de trabalho, o que comprovamos através dos efeitos proporcionados aos colegas de trabalho dele que eram menos pacientes a ponto de se deixar levar inúmeras advertências. Carlos tinha sim alguns problemas de saúde já relatados, mas conforme mostrado estes pioraram após passar por esta experiência.\\
        
        Tendo em vista todo o histórico da vida profissional de Carlos percebemos que um ambiente que contém um controle desnecessário em cima de seus funcionários assim como todas os outros pontos já elucidados na questão anterior pode provocar e precipitar transtornos mentais, uma vez que em todos os ambientes anteriores ele não havia sofrido acidentes ou tido problemas graves de saúde em decorrência dos mesmos, fatos que podem ser comprovados a partir dos resultados dos exames admissionais feitos anteriormente.
        
    \end{description}